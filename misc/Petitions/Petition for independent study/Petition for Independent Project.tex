\documentclass[12pt]{article}

\usepackage{geometry, setspace}

\geometry{letterpaper, margin = 1in}

\title{Petition for Independent Project}
\author{Anshul Singhvi}

\begin{document}

\maketitle

\doublespacing

This is a petition to take an independent project for the semester of Spring 2019.  The project is an effort to develop electronic analog neurons in continuation of work done previously in the Hastings lab.  Not only is this project relevant to my study as a 3-2 Physics major, it is also timebound; grant season is fast approaching, as is the April meeting of the New York State chapter of the American Physical Society, at which this research will be presented.

This project will build upon research conducted by several Simon's Rock students, most recently Rifah Tasnim, with a tangible final result planned - a system of interconnected analog neurons to simulate the muscle response of the \textit{C. elegans} worm.

As outlined in the faculty letter to the committee, the project will involve extensive practical work in the lab, as well as study of the FitzHugh-Nagumo equations and their dynamics.  To this end, I have already created an interactive simulator of FHN (FitzHugh-Nagumo) dynamics, and have helped to build and rescale one analog neuron.  I have had experience with circuits from high school and taking PHYS 101 (Physics II) with Mike Bergman in my second semester.  Additionally, I have also taken CMPT 260 (Scientific Computing) in Fall 2018, which gave me a lot of experience with numerical solution of differential equations as well as nonlinear dynamics and chaos theory.

As a student looking towards the Columbia 3-2 dual degree program in applied physics, this research project would not only give me valuable practical experience at the intersection of biology, physics and computing - arguably one of the more valuable fields of research going forward - it also gives me the opportunity to publish a paper, which would increase my hireability for internships, and an opportunity to present at the April APS meeting, giving me valuable opportunities for networking as well as giving the College more exposure in the sciences.


\end{document}
