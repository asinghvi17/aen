\RequirePackage{luatex85}

\documentclass[tikz]{standalone}

\usepackage[siunitx]{circuitikzgit}

\usepackage{xcolor}

\usetikzlibrary{calc, arrows.meta, backgrounds}

\tikzset{
  open socket/.style = {
    circle,
    fill = lightgray,
    inner sep = 1pt
  },
  filled socket/.style = {
    circle,
    fill = black,
    inner sep = 1pt
  }
}

% work with dipchips as well
\makeatletter\newlength{\Rlen}\setlength{\Rlen}{\pgf@circ@Rlen}\makeatother
\ctikzset{multipoles/dipchip/pin spacing=1}
\ctikzset{multipoles/external pins width=0.5}
\ctikzset{multipoles/dipchip/width=1}

\begin{document}

\begin{tikzpicture}[
    x=\Rlen,
    y=\Rlen,
  ]

  %% draw breadboard sections
  \foreach \y [evaluate = \y as \line using int(abs(\y))] in {0, ..., -13}{

    % draw line numbers off to the left
    \node [text=gray] at (-9, \y) {\line};
    \node [text=gray] at (9, \y) {\line};

    % draw bus and breadboard grid sockets
    \foreach \lab/\x in {l-/-8, lg/-7, A/-5, B/-4, C/-3, D/-2, E/-1, F/1, G/2, H/3, I/4, J/5, rg/7, r+/8} {
      \coordinate [open socket] (\lab\line) at (\x, \y);
    }

    % draw row connectivity for board proper
    \draw [lightgray, opacity = 0.2, very thick]
      (-5, \y) -- (-1, \y)
      (1, \y)  -- (5, \y);

  }

  % label breadboard columns

  \foreach \lab/\x in {A/-5, B/-4, C/-3, D/-2, E/-1, F/1, G/2, H/3, I/4, J/5} {
    \node [above of = \lab0] {\lab};
  }

  % label buses

  \node [above of=l-0] {V$_{-}$};
  \node [above of=lg0] {GND$_{}$};
  \node [above of=rg0] {GND$_{}$};
  \node [above of=r+0] {V$_{+}$};

  % draw bus connection lines
  \begin{scope} [
      on background layer  % these paths should not interfere with anything else
    ]
    \draw [very thick, red,  opacity = 0.2] (l-0) -- (l-13);
    \draw [very thick, blue, opacity = 0.2] (lg0) -- (lg13);

    \draw [very thick, blue, opacity = 0.2] (rg0) -- (rg13);
    \draw [very thick, red,  opacity = 0.2] (r+0) -- (r+13);
  \end{scope}

  %% We're done with the breadboard!
  %% Now, we need to draw the actual circuit :P

  % Now, I place the op amp:

  \node at (E1) (081) [
                        dipchip,
                        fill = cyan!20!white,
                        % circuitikz/multipoles/dipchip/pin spacing = 0.711,
                        % circuitikz/multipoles/dipchip/width = 1.03,
                        anchor = pin 1,
                        num pins = 8,
                        hide numbers
                    ] {TL081};

  % and fill its pins with tiny descriptions:

  \node [right, font=\tiny] at (081.bpin 1) {ON1};
  \node [right, font=\tiny] at (081.bpin 2) {IN-};
  \node [right, font=\tiny] at (081.bpin 3) {IN+};
  \node [right, font=\tiny] at (081.bpin 4) {$V_-$};

  \node [left,  font=\tiny] at (081.bpin 5) {ON2};
  \node [left,  font=\tiny] at (081.bpin 6) {OUT};
  \node [left,  font=\tiny] at (081.bpin 7) {$V_+$};
  \node [left,  font=\tiny] at (081.bpin 8) {NC};

  \foreach \i in {1, 2, ..., 8} {
    \node at (081.pin \i) [filled socket] {};
  }

  % Having placed the op amp, it now needs to be connected to voltage rails...

  \draw [yellow, thick] (l-4) coordinate [filled socket] () -- (A4) coordinate [filled socket] ();
  \draw [red,    thick] (r+2) coordinate [filled socket] () -- (J2) coordinate [filled socket] ();

  % Placing the voltage divider is a bit involved:

  \draw (lg2) to [R, l = $R_1$, a = 100<\kilo\ohm>, *-*] (B2);

  % I've taken a more circuituous route for this resistor, so that
  % it doesn't interfere with the op-amp to be placed there.
  \draw (D2) to ($(E0) + (0, 0.2)$) to [R, l = $R_1$, a = 100<\kilo\ohm>] ($(F0) + (0, 0.2)$) to (G2);
  % Because the path isn't direct, we need to set the black dots that
  % indicate connection points to the breadboard manually.
  \node [filled socket] at (D2) {};
  \node [filled socket] at (G2) {};

  % Having finished the voltage divider, it's important to indicate where signal lives:

  \coordinate [circle, fill = green!50!black, inner sep = 2pt] (signal) at (C2);
  \node at (signal) [below] {signal};

  % This concludes the ``nonlinear current device''.

  % Now, we construct the simulated inductor

  % beginning with the ``V-bus''.

  \draw [blue, thick] (B3) coordinate [filled socket] -- (B5) coordinate [filled socket];

  % and the bridge between the two sides of the breadboard.

  \draw [thick] (E5) coordinate [filled socket] -- (F5) coordinate [filled socket];

  % I now draw the membrane capacitor:

  \draw (lg5) to [C, l=$C_1$, a = 0.01<\micro\farad>, *-*] (A5);

  % and the bridging resistor to the V-bus:

  \draw (I3) to [R, l = $R_2$, a = 220<\kilo\ohm>, *-*] (I5);

  % Before any connectivity to it is established, I'll draw the second op amp (a dual one):

  \node at (E6) (082) [
                        dipchip,
                        fill = cyan!20!white,
                        % circuitikz/multipoles/dipchip/pin spacing = 0.711,
                        % circuitikz/multipoles/dipchip/width = 1.03,
                        anchor = pin 1,
                        num pins = 8,
                        hide numbers
                    ] {TL082};

  % and fill its pins with tiny descriptions:

  \node [right, font=\tiny] at (082.bpin 1) {1O};
  \node [right, font=\tiny] at (082.bpin 2) {1-};
  \node [right, font=\tiny] at (082.bpin 3) {1+};
  \node [right, font=\tiny] at (082.bpin 4) {$V_-$};

  \node [left,  font=\tiny] at (082.bpin 5) {2+};
  \node [left,  font=\tiny] at (082.bpin 6) {2-};
  \node [left,  font=\tiny] at (082.bpin 7) {2O};
  \node [left,  font=\tiny] at (082.bpin 8) {$V_+$};

  \foreach \i in {1, 2, ..., 8} {
    \node at (082.pin \i) [filled socket] {};
  }

  % This particular op amp has some associated coupling:

  \draw (G7) coordinate [filled socket] -- (G8) coordinate [filled socket];

  \draw (D6) coordinate [filled socket] -- (D7) coordinate [filled socket];

  % On the left of the board, the capacitor-resistor parallel pair live:

  \draw (H5) to [R, l = $R_3$, a = 100<\kilo\ohm>, *-*] (H7);
  \draw (J5) to [C, l = $C_2$, a = 0.47<\micro\farad>, *-*] (J9);

  % The resistor that connects the non-inverting input of op amp 2 to the output of op amp 1 (the bias voltage)
  % is a little complicated to draw directly.  Therefore, I'm going to use the same trick I used on the
  % voltage divider.

  \draw (G9) to ($(F10) - (0, 0.2)$) to [R, l = $R_4$, a = 1000<\kilo\ohm>] ($(E10) - (0, 0.2)$) to (C7);

  \draw (G9) coordinate [filled socket];
  \draw (C7) coordinate [filled socket];

  % I'm done with the bulk of the circuit!  Now, it's just the potentiometer left.

  \node (pot) at (E12) [european resistors, genericpotentiometershape, fill = cyan!20!white, rotate=90, anchor = wiper] {};

  \draw (pot.wiper) coordinate [filled socket] ();

  \draw (pot.right) -- (pot.right |- E11) -- (E11) coordinate [filled socket];

  \draw (pot.left) -- (pot.left |- E13) -- (E13) coordinate [filled socket];

  \draw [thick] (lg11) coordinate [filled socket] -- (A11) coordinate [filled socket];

  \draw [thick, yellow] (l-13) coordinate [filled socket] -- (A13) coordinate [filled socket];

  \draw (B12) coordinate [filled socket] -- (B8) coordinate [filled socket];


  \draw [yellow, thick] (l-9) coordinate [filled socket] () -- (A9) coordinate [filled socket] ();
  \draw [red,    thick] (r+6) coordinate [filled socket] () -- (J6) coordinate [filled socket] ();z

\end{tikzpicture}

\end{document}
