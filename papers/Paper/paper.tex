\documentclass[
    11pt,
    preprint,
    author-numerical,
    aps
]{revtex4-2}

\input{preamble.tex}

\tikzexternalize

\begin{document}

%==============================================================================%
%                              Title and abstract                              %
%==============================================================================%

\title{A minimal reaction-diffusion neural model generates {\emph{C. elegans}} undulation}
\author{Anshul Singhvi}
\affiliation{Bard College at Simon's Rock}
\date{\today}
\begin{abstract}
    The small (1 mm) nematode \emph{Caenorhabditis elegans} has become widely used as a model organism; in particular the \emph{C. elegans} connectome has been completely mapped, and \emph{C. elegans} locomotion has been widely studied (c.f. http://www.wormbook.org). We describe a minimal reaction-diffusion model for the \emph{C. elegans} central pattern generator (CPG) (c.f. Xu et al. 2018, Wen et al. 2012). We use simulation methods to show that a small network of FitzHugh (1961)-Nagumo (et al. 1962) neurons (one of the simplest neuronal models) can generate key features of \emph{C. elegans} undulation (c.f. Magnes et al. 2017) and thus locomotion. Compare the neuromechanical model of Izquierdo and Beer (2015). We also investigate dynamics and stability of the model.
\end{abstract}

\maketitle

\section{Introduction}\label{sec: intro}

hello world

\inputtikz{figures/fhn_dynamics/fhn_dynamics}


\section{Correspondence}\label{correspondence}
\makeatletter
Please direct all correspondence to hhastings@simons-rock.edu,
jemagnes@vassar.edu, or\\asinghvi17@simons-rock.edu.
\makeatother

\section{References}

\nocite{*}
\bibliography{References}
% \printbibliography[heading=none]
\end{document}
