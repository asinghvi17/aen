\documentclass[
    11pt,
]{article}

\usepackage{authblk}

% \usepackage{natbib}
% \setcitestyle{nature}

\usepackage[style=nature, natbib]{biblatex}

\addbibresource{References.bib}

\input{preamble.tex}

\tikzexternalize

\begin{document}
% \sffamily

%==============================================================================%
%                              Title and abstract                              %
%==============================================================================%

\title{A minimal reaction-diffusion neural model generates {\emph{C. elegans}} undulation}

\author[1]{Anshul Singhvi}

\author[1,3]{Harold Hastings}

\author[2]{Jennifer Magnes}

\author[2]{Cheris Congo}

\author[2]{Miranda Hulsey-Vincent}

\author[1]{Rifah Tasnim}

\author[1]{Naol Negassa}

\affil[1]{Bard College at Simon's Rock}
\affil[2]{Vassar College}
\affil[3]{Hofstra University}

\date{\today}

\begin{abstract}
    The small (1 mm) nematode \emph{Caenorhabditis elegans} has become widely used as a model organism; in particular the \emph{C. elegans} connectome has been completely mapped, and \emph{C. elegans} locomotion has been widely studied (c.f. http://www.wormbook.org \citet{corsi2015}). We describe a minimal reaction-diffusion model for the \emph{C. elegans} central pattern generator (CPG) \citet{xu2018,wen2012}. We use simulation methods to show that a small network of \citet{fitzhugh1955}-\citet{nagumo1962} neurons (one of the simplest neuronal models) can generate key features of \emph{C. elegans} undulation (see \citet{magnes2012}), and thus locomotion. Compare the neuromechanical model of \citet{izquierdo2018}. We also investigate dynamics and stability of the model.
\end{abstract}

\maketitle

\section{Introduction}\label{sec: intro}


The small (1 mm) nematode \emph{Caenorhabditis elegans} (\emph{C. elegans}) has become widely used as a model organism \citep{corsi2015}, and has been among the most widely studied biological models of neuronal development, locomotion and the central pattern generator \citep{katz2016}.
The C. elegans connectome has been completely mapped \citep{jabr} and, as described below, its locomotion has been widely studied.
``When crawling on a solid surface, the nematode C. elegans moves forward by propagating sinusoidal dorso-ventral retrograde contraction waves. A uniform propagating wave leads to motion that undulates about a straight line.'' \citep{kim2011}.
A different type of locomotion, often called swimming, occurs when nematodes are submerged in a liquid medium. The nematodes “switch” between these two gaits, under the regulation of particular serotonergic and dopaminergic neurons.

The purpose of this paper to describe a minimal reaction-diffusion model for the \emph{C. elegans} central pattern generator (CPG) \citep{xu2018, wen2012}. We use simulation methods to show that a small network of \citet{fitzhugh1955}-\citet{nagumo1962} neurons (one of the simplest neuronal models) based on a skeleton model of the C. elegans CPG can reproduce key features of C. elegans undulation \citep{magnes2012} and thus locomotion.

\section{The model central pattern generator}

The central pattern generator is a small neural circuit which regulates the movement of the nematode.  This structure is present in different forms in many animals, and it regulates many types of periodic motion.  \citet{xu2018} proposed an architecture for the CPG of \emph{C. elegans} which is described below in \fref{fig: xu_cpg}.

\begin{figure}[h!]
    \label{fig: xu_cpg}
    \centering
    \includegraphics[width=7cm]{figures/xu_cpg/xu_cpg.png}
    \caption{Pirated from Xu}
\end{figure}

The central pattern generator has two principal components.  First is the \textbf{head oscillator}. Described by \citet{gjorgjieva2014}, the head oscillator consists of two “head neurons” with mutually inhibitory coupling.  Oscillations are generated when this coupling destabilizes an excitable steady state.

Second is the \textbf{descending pathway}, which consists of pairs of coupled dorsal and ventral neurons.  These follow the body of the worm, and are linked to motor neurons and muscles.

While \textit{C. elegans} has twelve pairs of motor neurons, we have only used six pairs in our model.  \fref{fig: cpg} is a depiction of our simplified model as a graph, wherein neurons are nodes, and the arrows between them symbolize connections.

\begin{figure}[h!]
    \label{fig: cpg}
    \centering
    \inputtikz{figures/cpg/cpg}
    \caption{Our simplified central pattern generator model.  Descending coupling (shows inhibitory connections, and potential flows through gap junctions, not necessarily symmetric)}
\end{figure} % TODO: legend

\section{The FitzHugh-Nagumo Neuron}\label{sec: fhn}

In accord with the goal of this paper, we sought the simplest relevant neuronal model.  The classical Hodgkin-Huxley\cite{hodgkin1952} model of squid neurons has led to a variety of simpler conduction models, including the Morris-Lecar\cite{morris1981} and Fitzhugh-Nagumo models.

The FHN model consists of two dynamical variables; a fast activator variable $v$ corresponding to the (rescaled) membrane potential, and a slow inhibitor variable $w$ corresponding to a generalized gating variable.

\begin{equation}
    \label{eq: fhn}
    \begin{aligned}
        dv &= f(v) + w - I_\mathrm{ext}\\
        dw &= ϵ(a - bv)\\
        f(v) &= \frac{v^3}{3} - v
    \end{aligned}
\end{equation}

In this system, $f(v)$ can be any function which retains the appropriate dynamics.  In our analog implementation, we use a piecewise linear approximation to the cubic, in order to simplify the circuit and avoid using expensive components.

Xu et al. used a simplified two-variable model consisting of a fast, cubic-like activator variable (see the V-nullcline) and a slow, non-linear inhibitor variable (see the n-nullcline). Both the Morris-Lecar model and the Fitzhugh-Nagumo model have similar activator nullclines.

\begin{figure}[h!]
    \label{fig: nm nullclines}
    \centering
    \inputtikz{figures/neuron_dynamics/neuron_dynamics}
    \caption{Nullclines of several different neuronal models; on the right is the biological model of Xu et al, in the centre is the Morris-Lecar model, and to the right is the FitzHugh-Nagumo model.  These have been arranged in order of decreasing complexity.}
\end{figure} % TODO: legend

The original system was meant to model one neuron only.  We use diffusion to
model a synapse. A positive coefficient would simulate a gap junction or
an excitatory synapse; a negative coefficient would simulate inhibitory
coupling \citep{collins1994}.

The equations, when modified for synaptic connections, look like this:

\begin{equation}
    \label{eq: fhnd}
    \begin{aligned}
        dv &= f(v) + w - I_\mathrm{ext} + D(Δv)\\
        dw &= ϵ(a - bv)\\
        f(v) &= \frac{v^3}{3} - v
    \end{aligned}
\end{equation}

\begin{figure}[h!]
    \label{fig: fhn_dynamics}
    \centering
    \inputtikz{figures/fhn_dynamics/fhn_dynamics}
    \caption{The nullclines of the FitzHugh-Nagumo neuron.  Nullclines are isoclines where the derivative of a variable is zero.  Here we show an oscillatory mode and an excitable one.}
\end{figure} % TODO: legend



\section{Simulation}

The simulation was performed in Python, using the standard SciPy ODE solvers.

\subsection{Methods}

See the appendix, or attached code, for how we simulated this motion.  It should probably also be published on Github - I could make a Jupyter notebook with it.

\subsection{Undulation}

We achieved the undulatory effect by some Gaussian smoothing and B-splines to simulate the effect of real muscles.

\subsection{Comparison to real worm}

Cite the paper which Jenny sent here.  It can be seen that our model closely approximates an unconstrained worm, specifically of the wild type.

While the angle which a worm crawling on agar subtends tends to be oblique, as the constraint on the worm decreases, the angle becomes more and more acute - consider the videos . It is not hard to see that a fully unconstrained worm might behave much like our model does.

\section{Analog implementation}

- Nagumo circuit, disadvantages

- Keener improved, op amps, why is that good - scalability, etc

- Our modifications - diffusion, coupling mechanism.  Some mention of mathematics involved

- Figures: circuits, circuit nullclines,

\section{Comparison of analog and simulation}

For this section, we will focus on the timeseries output of the neurons, and not on the end worm.  Include the relevant figure here - you can see clearly that the effect is the same.  There are some differences in the waveform because of the different activation function in the circuit (linear interpolation).

\section{Conclusion}

Mention the bullet points from the presentation.  How could a system like this be used in applications?  End with some future paths.

\section{Correspondence}\label{correspondence}
\makeatletter
Please direct all correspondence to hhastings@simons-rock.edu,
jemagnes@vassar.edu, or\\asinghvi17@simons-rock.edu.
\makeatother

\section{References}

% \nocite{*}
% \bibliography{References}
\printbibliography[heading=none]
\end{document}
